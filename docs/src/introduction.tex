% vim: set tw=78 sts=2 sw=2 ts=8 aw et ai:

In the circumstances of a disaster (either natural or caused by man), there is
generally a high level of confusion among large groups of people. The
immediate action that needs to be taken is usually finding a safe spot to
hide. Afterwards, one can attempt to eliminate the cause of the disaster and
rescue the people left behind.

Because of the general confusion, the immediate decision to escape may be based
only on intuition, and thus is not optimal.

This project's goal is to create a mobile application that helps the user in an
emergency situation by pointing to the best safe place to hide. The list of
alternatives for shelter can either be crowdsourced from the other
participants to the disaster, or provided by the owner of the venue themselves
(say for example an accident happens during a concert or a sports game).

The application has a component which is permanently running in the background
as a service, and keeps listening for events that might trigger an emergency
alert.

Once emergency state is triggered, the active part of the application, in the
form of an activity, is automatically brought to the user's foreground, and
guides them to the best known safe spot.

While being a scenario that lends itself well to a mesh network approach, the
proposed prototype follows a traditional approach with a single, centralized
server. This is responsible for receiving messages from all mobile clients,
either sharing their current location, or the location of a newfound shelter.

The server keeps track of the positions of all clients and safe spots. On
request, it will apply a decision-making algorithm and select the most
appropriate safe location for the client who is asking.

The client is able to customize the way safe locations are being selected by
the server. As part of the decision-making algorithm, the user's perceived
importance of the following criteria is taken into consideration:

\begin{itemize}
  \item Proximity to the respective exit point. If this is perceived as
    important to the user, the server will tend to prefer guiding this user
    towards exit points that are close to him, with a higher precedence over
    other criteria.
  \item Safety of the exit point. This can be rephrased as "guarantee that
    the user can indeed exit through that point". A simple way to measure this
    (negatively) is to count the number of people who are in the vicinity of
    that exit spot, but cannot pass through it.
  \item How crowded is the exit point. The metric proposed here is the speed
    that people in the exit's vicinity are able to pass through it. A
    congested exit point permits people to exit at a much lower rate than one
    that is free. The difference between the metric used for crowdedness and
    the one used for safety is the fact that crowdedness only looks at the
    people who managed to escape, while safety looks at those who didn't.
  \item Being close to friends.
\end{itemize}
